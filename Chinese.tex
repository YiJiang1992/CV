\documentclass[11pt]{moderncv}
\moderncvstyle[roman]{casual}
%\usepackage{ctex}
\usepackage{CJKutf8}
%\usepackage{fontspec}

\phone{+86-186-0865-1884} \email{JY369356904@163.com}
\address{Northeastern University, Shenyang, China}{}
\def\ConTeXt{%
  C%
  \kern-.0333emo%
  \kern-.0333emn%
  \kern-.0667em\TeX%
  \kern-.0333emt}

\newcommand{\listitem}[1]{\textbullet\hspace{1em}\parbox[t]{13cm}{#1}\vspace{0.5em}}

\definecolor{see}{rgb}{0.5,0.5,0.5}
\newcommand{\up}[1]{\ensuremath{^\textsf{\scriptsize#1}}}
\newcounter{count}



\begin{document}

\begin{CJK*}{UTF8}{gbsn}

% \firstname{姜} \familyname{艺}
 \cnname{\hspace{0.2em}\color{blue}{姜 艺}}
 \maketitle
%\makequote
\vspace{-12mm}
%\rule{17cm}{0.4mm}
%\section{Contact Information}

\parbox[t]{2.5cm}{
                \textbf{地址}\\
                \textbf{}\\
                \textbf{}\\
                \textbf{籍贯}\\
                \textbf{出生年月}\\
                %\textbf{英语水平}\\
                \textbf{}}
\parbox[t]{7cm}{沈阳市和平区文化路3巷11号东北大学\\
                流程工业综合自动化国家重点实验室\\
                流程工业综合自动化国际合作联合实验室\\
                湖北省鄂州市\\
                1992.08\\
                %男\\

                %四级(CET4):470\\
                %六级(CET6):445}
                }
\parbox[t]{2cm}{
               \textbf{联系电话}\\
               \textbf{E-mail}\\
               \textbf{政治面貌}\\
               %\textbf{任职}\\
               %\textbf{学术任职}\\
               %\textbf{技能证书}\\
               \textbf{民族}\\
               \textbf{个人主页}\\
               }
\hspace{.1em}
\parbox[t]{5cm}{
               +86-186-0865-1884\\
               JY369356904@163.com\\
               中共党员\\
               %班长\\
               %Student Member, IEEE\\
               %计算机二级(C语言)\\
               %机动车驾驶证(C1)
               汉族\\
               yijiang1992.github.io\\
               }
\vspace{3mm}
\section{经历}
\cventry{2018/3-2019/3}{阿尔伯塔大学}{}{University of Alberta (UA)}{Edmonton, Alberta, Canada} {}
\cvline{$\bullet$}{\textbf{助理研究员}}
\cvline{$\bullet$}{\textbf{导师:} Tongwen Chen 教授(IEEE Fellow, IFAC Fellow, 加拿大工程院院士)}
\cvline{$\bullet$}{\textbf{研究实验室:}  电子与计算机工程学院}
\cventry{2017/1-2017/7}{美国德克萨斯大学阿灵顿分校}{}{University of Texas at Arlington (UTA)}{Arlington, TX, USA} {}
\cvline{$\bullet$}{\textbf{访问学者}}
\cvline{$\bullet$}{\textbf{导师:} Frank L. Lewis 教授(IEEE Fellow, IFAC Fellow, 美国发明家科学院院士)}
\cvline{$\bullet$}{\textbf{研究实验室:} 美国德克萨斯大学阿灵顿分校研究所(UTARI)}

\section{教育经历}
\cventry{2014/9-2020/4}{东北大学}{}{Northeastern University (NEU), ``985''工程, ``211''工程}{中国沈阳} {}
\cvline{$\bullet$}{\textbf{控制理论与控制工程专业硕博连读}, 推荐免试攻读}
\cvline{$\bullet$}{\textbf{导师:} 柴天佑 教授(IEEE Fellow, IFAC Fellow, 中国工程院院士, 欧亚科学院院士, 英国皇家工程院Distinguished Visiting Fellowship, 日本学术振兴协会 Invitation Fellowship)}
\cvline{$\bullet$}{\textbf{研究实验室:} 流程工业综合自动化国家重点实验室, 流程工业综合自动化国际合作联合实验室}
\cventry{2010/9-2014/6}{东北大学}{}{Northeastern University (NEU), ``985''工程, ``211''工程}{中国沈阳} {}
\cvline{$\bullet$}{\textbf{自动化专业学士}, 信息科学与工程学院}


\section{研究经历}
%
\cventry{2018/3-至今}{事件触发控制方法}{}{电子与计算机工程学院}{加拿大阿尔伯塔大学} {}
\cvline{$\bullet$}{\textbf{导师:} Tongwen Chen 教授(IEEE Fellow, IFAC Fellow, 加拿大工程院院士)}
\cvline{$\bullet$}{\textbf{研究内容:} 算法研究}
%
\cventry{2017/1-至今}{基于强化学习的工业过程运行优化控制方法}{}{美国德克萨斯大学阿灵顿分校研究所}{美国德克萨斯大学阿灵顿分校} {}
\cvline{$\bullet$}{\textbf{导师:} Frank L. Lewis 教授(IEEE Fellow, IFAC Fellow, 美国发明家科学院院士)}
\cvline{$\bullet$}{\textbf{研究内容:} 算法研究}
%
\cventry{2013/6-至今}{双网环境下工业过程运行优化控制方法}{}{流程工业综合自动化国家重点实验室, 流程工业综合自动化国际合作联合实验室}{东北大学} {}
\cvline{$\bullet$}{\textbf{导师:} 柴天佑 教授(IEEE Fellow, IFAC Fellow, 中国工程院院士, 欧亚科学院院士, 英国皇家工程院Distinguished Visiting Fellowship, 日本学术振兴协会 Invitation Fellowship)}
\cvline{$\bullet$}{\textbf{研究内容:} 网络化运行控制方法, 无线控制系统研发}
%
\cventry{2013/8-2013/12}{基于视觉的四旋翼目标跟踪检测研究}{}{信息科学与工程学院人工智能与机器人研究所}{东北大学} {}
\cvline{$\bullet$}{\textbf{导师:} 张云洲 教授}
\cvline{$\bullet$}{\textbf{研究内容:} 四旋翼视觉图像处理、控制算法研究}
%
\cventry{2012/10-2013/8}{飞思卡尔智能车}{}{飞思卡尔智能车实验室}{东北大学} {}
\cvline{$\bullet$}{\textbf{导师:} 陈述平 教授, 闻时光 讲师}
\cvline{$\bullet$}{\textbf{研究内容:} 智能车硬件系统搭建, 循迹、控制算法研究}
%
\cventry{2011/10-2013/5}{基于视觉的罪犯搜捕系统}{}{信息科学与工程学院自动化研究所}{东北大学}{}
\cvline{$\bullet$}{\textbf{导师:} 贾明兴 教授}
\cvline{$\bullet$}{\textbf{研究内容:} 人脸检测与识别}
%
\cventry{2011/10-2012/10}{``牛牛''机器人足球竞赛研究}{}{信息科学与工程学院人工智能与机器人研究所}{东北大学} {}
\cvline{$\bullet$}{\textbf{导师:} 佟国峰 教授}
\cvline{$\bullet$}{\textbf{研究内容:} 机器人视觉图像处理}

\section{项目}

\cventry{2018.01-2019.12}{浮选过程运行控制方法}{东北大学博士生科研创新项目}{经费4万}{主持者}{}{}
\cventry{2016.01-2020.12}{面向智慧企业的工业认知网络体系架构、设计方法与应用验证}{国家自然科学基金重点项目(No. 61533015)}{经费380.5万}{参与者}{}{}
\cventry{2014.01-2018.12}{双网环境下工业生产过程运行控制与优化控制方法研究}{国家自然科学基金重点项目(No. 61333012)}{经费300万}{参与者}{}{}
\cventry{2014.01-2016.12}{基于无线传感器网络的工业过程运行反馈控制方法研究}{国家自然科学基金青年科学基金项目(No.61304028)}{经费25万元}{参与者}{}{}

\section{发表论文与专利}

%\cventry{\textbf{论文}}{以第一作者$\backslash$通讯作者$\backslash$第一学生作者发表}{}{}{}{}

\cvline{}{\textbf{期刊论文}}\setcounter{count}{1}
\cvline{[\thecount]}{\textbf{Yi Jiang}, Dawei Shi, Jialu Fan$^*$, Tianyou Chai, Tongwen Chen. Set-Valued Feedback Control and its Application to Event-Triggered Sampled-Data Systems. \textbf{IEEE Transactions on Automatic Control}, to be published.}\addtocounter{count}{1}
\cvline{[\thecount]}{Wenqian Xue, Jialu Fan$^*$, Victor Gabriel Lopez Mejia, Jinna Li$^*$, \textbf{Yi Jiang}, Tianyou Chai, Frank L. Lewis. New Methods for Optimal Operational Control of Industrial Processes using Reinforcement Learning on Two Time-Scales. \textbf{IEEE Transactions on Industrial Informatics}, 2020, 16(5): 3085-3099. (EI\&SCI)}\addtocounter{count}{1}
\cvline{[\thecount]}{庞文砚, 范家璐*, \textbf{姜艺}, 刘易斯·弗兰克. 基于强化学习的部分线性离散时间系统最优输出调节. 自动化学报, 已录用. (EI)}\addtocounter{count}{1}
\cvline{[\thecount]}{Jialu Fan, Qian Wu, \textbf{Yi Jiang}$^*$, Tianyou Chai, Frank L. Lewis. Model-Free Optimal Output Regulation for Linear Discrete-Time Lossy Networked Control Systems. IEEE Transactions on Systems, Man, and Cybernetics: Systems, to be published, DOI: 10.1109/TSMC.2019.2946382. (EI\&SCI)}\addtocounter{count}{1}
\cvline{[\thecount]}{ 李臻, 范家璐*, \textbf{姜艺}, 柴天佑. 一种基于Off-policy的无模型输出数据反馈$H_\infty$控制方法. 自动化学报, 已录用. (EI)}\addtocounter{count}{1}
\cvline{[\thecount]}{\textbf{Yi Jiang}, Bahare Kiumarsi, Jialu Fan$^*$, Tianyou Chai, Jinna Li, Frank L. Lewis. Optimal Output Regulation of Linear Discrete-time Systems with Unknown Dynamics using Reinforcement Learning. \textbf{IEEE Transactions on Cybernetics}, to be published, DOI: 10.1109/TCYB.2018.2890046. (EI\&SCI)}\addtocounter{count}{1}
\cvline{[\thecount]}{吴倩, 范家璐$^*$, \textbf{姜艺}, 柴天佑. 无线网络环境下数据驱动混合选别浓密过程双率控制方法. \textbf{自动化学报}, 2019, 45(6): 1128-1141. (EI)}\addtocounter{count}{1}
\cvline{[\thecount]}{\textbf{Yi Jiang}, Jialu Fan$^*$, Tianyou Chai, Frank L. Lewis. Dual-Rate Operational Optimal Control for Flotation Industrial Process with Unknown Operational Model. \textbf{IEEE Transactions on Industrial Electronics}, 2019, 66(6): 4587-4599. (EI\&SCI)}\addtocounter{count}{1}
\cvline{[\thecount]}{Jinna Li, Tianyou Chai$^*$, Frank L. Lewis, Zhengtao Ding, \textbf{Yi Jiang}. Off-Policy Interleaved $Q$-Learning: Optimal Control for Affine Nonlinear Discrete-Time Systems. \textbf{IEEE Transactions on Neural Networks and Learning Systems}, 2019, 30(5): 1308-1320. (EI\&SCI)}\addtocounter{count}{1}
\cvline{[\thecount]}{\textbf{姜艺}, 范家璐$^*$, 贾瑶, 柴天佑. 数据驱动的浮选过程运行反馈解耦控制方法. \textbf{自动化学报}, 2019, 45(4): 759-770. (EI)}\addtocounter{count}{1}
\cvline{[\thecount]}{Xinglong Lu, Bahare Kiumarsi, Tianyou Chai$^*$, \textbf{Yi Jiang}, Frank L. Lewis. Operational Control of Mineral Grinding Processes Using Adaptive Dynamic Programming and Reference Governor. \textbf{IEEE Transactions on Industrial Informatics}, 2019, 15(4): 2210-2221. (EI\&SCI)}\addtocounter{count}{1}
\cvline{[\thecount]}{\textbf{Yi Jiang}, Jialu Fan$^*$, Tianyou Chai, Frank L. Lewis, Jinna Li. Tracking Control for Linear Discrete-time Networked Control Systems with Unknown Dynamics and Dropout. \textbf{IEEE Transactions on Neural Networks and Learning Systems}, 2018, 29(10): 4607-4620. (EI\&SCI)}\addtocounter{count}{1}
\cvline{[\thecount]}{\textbf{姜艺}, 李砚浓, 范家璐$^*$. 基于模型预测控制的工业过程设定点调整方法. \textbf{控制工程}, 2018, 25(6): 980-984.}\addtocounter{count}{1}
\cvline{[\thecount]}{\textbf{Yi Jiang}, Jialu Fan$^*$, Tianyou Chai, Jinna Li, Frank L. Lewis. Data-Driven Flotation Industrial Process Operational Optimal Control Based on Reinforcement Learning. \textbf{IEEE Transactions on Industrial Informatics}, 2018, 14(5): 1974-1989.  (EI\&SCI)}\addtocounter{count}{1}
\cvline{[\thecount]}{Jialu Fan$^*$, \textbf{Yi Jiang}, Tianyou Chai. MPC-Based Setpoint Compensation with Unreliable Wireless Communications and Constrained Operational Conditions. \textbf{Neucomputing}, 2017, 270: 110-121. (EI\&SCI)}\addtocounter{count}{1}
\cvline{[\thecount]}{范家璐$^*$, \textbf{姜艺}, 柴天佑. 无线网络环境下工业过程运行反馈控制方法. \textbf{自动化学报}, 2016, 42(8): 1166-1174. (EI)}\addtocounter{count}{1}
\cvline{[\thecount]}{李砚浓, 李汀兰, \textbf{姜艺}, 范家璐$^*$. 基于RBF神经网络自适应PID四旋翼飞行器控制. \textbf{控制工程}, 2016, 23(3): 378-382.}\addtocounter{count}{1}

\cvline{}{\textbf{会议论文}}\setcounter{count}{1}
\cvline{[\thecount]}{Jialu Fan$^*$, Wenkuan Feng, \textbf{Yi Jiang}. Operational Feedback Control of Industrial Process Under Wireless Packet Disordering. \textbf{2019 International Conference on Advanced Mechatronic Systems (ICAMechS)}. 2019: 87-91. (EI)}\addtocounter{count}{1}
\cvline{[\thecount]}{Jialu Fan$^*$, Zhen Li, \textbf{Yi Jiang}, Tianyou Chai, Frank L. Lewis. Model-Free Linear Discrete-Time System $H_\infty$ Control Using Input-Output Data. \textbf{2018 International Conference on Advanced Mechatronic Systems (ICAMechS)}. 2018: 207-212. (EI)}\addtocounter{count}{1}
\cvline{[\thecount]}{\textbf{Yi Jiang}, Jialu Fan$^*$, Tianyou Chai, Tongwen Chen. Setpoint Dynamic Compensation via Output Feedback Control with Network Induced Time Delays. \textbf{Proceeding of the American Control Conference, Chicago}, 2015: 5384-5389. (EI)}\addtocounter{count}{1}
\cvline{[\thecount]}{\textbf{Yi Jiang}; Jialu Fan$^*$; Yewei Zhang; Lei Wang. An Experimental Study on Integrated Network-based Operational Control Method. \textbf{Mechatronics and Control (ICMC), 2014 1st International Conference on. IEEE,} 2014: 1995-2000. (EI)}\addtocounter{count}{1}
\cvline{[\thecount]}{Yannong Li, \textbf{Yi Jiang}, Yutao Fu, Jialu Fan$^*$. Prediction-Prevention Mathematic Model of Ebola's Spread and
Eradication. \textbf{Proceeding of the 2015 IEEE International Conference on Information and Automation (ICIA)}, 2015: 1774-1779. (EI)}\addtocounter{count}{1}
\cvline{[\thecount]}{Yewei Zhang, Jialu Fan$^*$, \textbf{Yi Jiang}, Lei Wang. Semi-physical Simulation Platform for Double Layer Network-Based Operational Control. \textbf{Intelligent Control and Automation (WCICA), 2014 11th World Congress on. IEEE}, 2014: 1118-1123. (EI)}\addtocounter{count}{1}
\cvline{[\thecount]}{Mingxing Jia$^*$, Junqiang Du, Tao Cheng, Ning Yang, \textbf{Yi Jiang}, Zhen Zhang. An Improved Detection Algorithm of Face with Combining Adaboost and SVM. \textbf{Control and Decision Conference (CCDC), 2013 25th Chinese. IEEE}, 2013: 2459-2463. (EI)}\addtocounter{count}{1}
%\cventry{}{与他人合作发表}{}{}{}{}

%\cvline{}{\textbf{期刊论文:}}\setcounter{count}{1}



%\cvline{}{\textbf{会议论文:}}\setcounter{count}{1}



%\cventry{\textbf{专利}}{以第一作者或第一学生作者发表}{}{}{}{}\setcounter{count}{1}
\cvline{}{\textbf{专利}}\setcounter{count}{1}
\cvline{[\thecount]}{范家璐, \textbf{姜艺}, 柴天佑. 一种双网环境下浮选工业过程运行控制系统及方法. 中国: 2014101756.32. (授权时间: 2016/03/04)}\addtocounter{count}{1}
\cvline{[\thecount]}{范家璐, 刘锐, \textbf{姜艺}, 柴天佑. 一种基于无线通信的单容水箱运行控制系统及方法. 中国: 201510776737.7. (授权时间: 2017/12/29)}\addtocounter{count}{1}
%\cventry{}{与他人合作发表}{}{}{}{}\setcounter{count}{1}


%

\section{服务}
%

\cvline{}{\textbf{期刊论文服务:}}\setcounter{count}{1}
\cvline{[\thecount]}{Associate Editor of Advanced Control for Applications: Engineering and Industrial Systems (2020-present)}\addtocounter{count}{1}


\cvline{}{\textbf{期刊论文审稿:}}\setcounter{count}{1}
\cvline{[\thecount]}{IEEE Transactions on Automatic Control}\addtocounter{count}{1}
\cvline{[\thecount]}{Automatica}\addtocounter{count}{1}
\cvline{[\thecount]}{IEEE Transactions on Neural Networks and Learning Systems}\addtocounter{count}{1}
\cvline{[\thecount]}{IEEE Transactions on Systems, Man and Cybernetics: Systems}\addtocounter{count}{1}
\cvline{[\thecount]}{International Journal of Robust and Nonlinear Control}\addtocounter{count}{1}
\cvline{[\thecount]}{Industrial \& Engineering Chemistry Research}\addtocounter{count}{1}
\cvline{[\thecount]}{Neurocomputing}\addtocounter{count}{1}
\cvline{[\thecount]}{Optimal Control, Applications and Methods}\addtocounter{count}{1}
\cvline{[\thecount]}{Wireless Communications and Mobile Computing}\addtocounter{count}{1}
\cvline{[\thecount]}{控制理论与应用}\addtocounter{count}{1}
\cvline{[\thecount]}{应用科学学报}\addtocounter{count}{1}
\cvline{[\thecount]}{控制工程}\addtocounter{count}{1}

\cvline{}{\textbf{会议服务:}}\setcounter{count}{1}
\cvline{[\thecount]}{2018 IEEE International Conference on Information and Automation (ICIA)程序委员会参与者}\addtocounter{count}{1}
\cvline{[\thecount]}{2017中国过程控制会议分会场副主席}\addtocounter{count}{1}

\cvline{}{\textbf{会议论文审稿:}}\setcounter{count}{1}
\cvline{[\thecount]}{IEEE Conference on Decision and Control (CDC)}\addtocounter{count}{1}
\cvline{[\thecount]}{American Control Conference (ACC)}\addtocounter{count}{1}
\cvline{[\thecount]}{Chinese Control Conference (CCC, 中国控制会议) }\addtocounter{count}{1}
\cvline{[\thecount]}{World Congress on Intelligent Control and Automation (WCICA)}\addtocounter{count}{1}
\cvline{[\thecount]}{Chinese Control and Decision Conference (CCDC, 中国控制与决策会议)}\addtocounter{count}{1}
\cvline{[\thecount]}{Data Driven Control and Learning Systems Conference (DDCLS)}\addtocounter{count}{1}
\cvline{[\thecount]}{IEEE International Conference on Information and Automation (ICIA)}\addtocounter{count}{1}
%

%
\cvline{}{\textbf{会员:}}\setcounter{count}{1}
\cvline{[\thecount]}{IEEE Student Member, 2014-2020}\addtocounter{count}{1}
\cvline{[\thecount]}{IEEE Member, 2020-present}\addtocounter{count}{1}
%

\section{学术会议经历}

\cventry{2019/7}{The 1st International Conference on Industrial Artificial Intelligence}{}{July 23 - 25, 2019}{Shenyang, Liaoning, China}{}{}
\cventry{2019/5}{The 3rd International Symposium on Autonomous Systems}{}{May 29-31, 2019}{Shanghai, China}{}{}
\cventry{2017/7}{第28届中国过程控制会议}{}{2017 年}{7月29日至8月7日, 中国重庆}{}{}
\cventry{2016/11}{第31届中国自动化学会青年学术年会}{}{2016年}{11月11日至11月12日, 中国武汉}{}{}
\cventry{2015/11}{第4届中国自动化大会}{}{2015年}{11月27日至11月29日, 中国武汉}{}{}
\cventry{2015/8}{第26届中国过程控制会议}{}{2015 年}{7月31日至8月3日, 中国南昌}{}{}
\cventry{2015/7}{The 2015 American Control Conference}{}{July 1-3, 2015}{Chicago, America}{}
\cventry{2014/8}{第25届中国过程控制会议}{}{2014 年}{8月9日至11日, 中国大连}{}{}
\cventry{2014/7}{The 1st International Conference on Mechatronics and Control}{}{July 3-5, 2014}{Jinzhou, China}{}
\cventry{2014/6}{The 11th World Congress on Intelligent Control and Automation}{}{June 29-July 4, 2014}{Shenyang, China}{}

\section{奖学金}

\cventry{2018/9}{博士研究生国家奖学金}{}{}{}{}{}
\cventry{2018/7}{东北大学卓越博士奖学金}{}{}{}{}{}
\cventry{2018/5}{张嗣瀛教育基金优秀博士生奖学金}{}{}{}{}{}
\cventry{2015/10}{东北大学硕士研究生国家奖学金}{}{}{}{}{}
\cventry{2015/9}{东北大学研究生学业奖学金一等奖}{}{}{}{}{}
\cventry{2014/9}{东北大学研究生学业奖学金一等奖}{}{}{}{}{}
\cventry{2013-2014}{2013-2014第一学期东北大学二等奖学金}{}{}{}{}{}
\cventry{2012-2013}{2012-2013第二学期东北大学二等奖学金}{}{}{}{}{}
\cventry{2012-2013}{2012-2013第一学期东北大学二等奖学金}{}{}{}{}{}
\cventry{2011-2012}{2011-2012第二学期东北大学三等奖学金}{}{}{}{}{}
\cventry{2011-2012}{2011-2012第一学期东北大学二等奖学金}{}{}{}{}{}
\cventry{2010-2011}{2010-2011第二学期东北大学二等奖学金}{}{}{}{}{}

\section{荣誉}

\cventry{2019}{东北大学信息科学与工程学院优秀共产党员}{}{}{}{}{}
\cventry{2019}{东北大学“五四青年奖章”十佳研究生}{}{}{}{}{}
\cventry{2018}{沈阳市优秀研究生}{}{}{}{}{}
\cventry{2018}{东北大学三好研究生}{}{}{}{}{}
\cventry{2017}{东北大学流程工业综合自动化国家重点实验室优秀科研奖}{}{}{}{}{}
\cventry{2016}{东北大学流程工业综合自动化国家重点实验室优秀科研奖}{}{}{}{}{}
\cventry{2017}{2017年中国冶金教育学会优秀硕士学位论文}{}{}{}{}{}
\cventry{2016}{2016级辽宁省优秀研究生毕业论文}{}{}{}{}{}
\cventry{2016}{2016级东北大学优秀研究生毕业论文}{}{}{}{}{}
\cventry{2016}{日本``樱花科技''青年交流计划}{}{}{}{}{}
\cventry{2015}{沈阳市优秀研究生}{}{}{}{}{}
\cventry{2015}{2014-2015学年东北大学三好研究生}{}{}{}{}{}
\cventry{2014/9}{2014级东北大学优秀本科毕业论文}{}{}{}{}{}
\cventry{2014}{东北大学创新学分奖励(12学分)}{}{}{}{}{}
\cventry{2012-2013}{2012-2013学年东北大学优秀团员}{}{}{}{}{}
\cventry{2012-2013}{2012-2013学年东北大学科技创新优秀个人}{}{}{}{}{}
\cventry{2011-2012}{2011-2012学年东北大学优秀团员}{}{}{}{}{}

\section{获奖}

\cventry{2018/9}{Best Paper Award in 2018 International Conference on Advance Mechatronic Systems (ICAMechS)}{}{}{}{}{}
\cventry{2018/7}{第29届中国过程控制会议优秀学生论文奖}{}{}{}{}{}
\cventry{2013/12}{2013年全国大学生电子设计大赛二等奖}{}{}{}{}{}
\cventry{2013/12}{2013年``TI''杯辽宁省高等学校本科大学生电子设计大赛一等奖}{}{}{}{}{}
\cventry{2013/7}{第八届飞思卡尔全国大学生智能车竞赛东北赛区二等奖}{}{}{}{}{}
\cventry{2013/2}{2013年美国数学建模大赛国际一等奖}{}{}{}{}{}
\cventry{2012/9}{2012年全国大学生数学建模竞赛辽宁省三等奖}{}{}{}{}{}
\cventry{2012/8}{2012年``TI''杯辽宁省普通高等学校本科大学生电子设计大赛二等奖}{}{}{}{}{}
\cventry{2012/6}{首届``ICM''杯国际大学生数学建模邀请赛中国区分赛三等奖}{}{}{}{}{}
\cventry{2011/12}{第三届全国大学生数学竞赛辽宁省一等奖}{}{}{}{}{}
\cventry{2011/6}{东北大学机械制图竞赛二等奖}{}{}{}{}{}

\section{技能}
\cvline{编程语言}{C, C++, C\#, JAVA, C51, Embedded C for Freescale Kinetis, 汇编语言(Assembly Language), MATLAB, VHDL, OpenCV, \LaTeX{}}

\cvline{软件}{Step 7, WinCC, Altium Designer 9, Protel 99 SE, CAXA, Solidworks, Microsoft Office, SPSS, Keil, IAR Embedded Workbench, Code Composer Studio, CodeWarrior, Visual Studio, Multisim}

\nocite{*}
\bibliographystyle{plain}
\bibliography{jdoe_publications}
\end{CJK*}
\end{document}